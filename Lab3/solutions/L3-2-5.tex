There are 4 bytes dedicated to fragmentation as part of IPv4 and 8 bytes as part of IPv6. When it comes to overhead, IPv4 has a significant overhead when not using the "Don't fragment" header appropriately. This forces intermediate routers to fragment IP datagrams, resulting in a large CPU overhead on those nodes. In case of IPv6 and IPv4 using the "Don't Fragment", there is overhead caused by the path MTU discovery. Depending on the size of the path, a number of ICMP messages have to be sent to the source. Furthermore, when a single fragment is dropped, the entire upper-level (TCP or UDP) packet has to be retransmitted as well.