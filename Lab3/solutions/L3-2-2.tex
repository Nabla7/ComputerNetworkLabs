The path MTU value is cached to ensure that this value has to be computed and coordinated just once. For IPv6, if the cache already contained a value of 1280, H1 could have fragmented the datagram in advance. In other words, if we did not clear the cache before running the experiment a second time, no ICMP message would be visible. In case of IPv4, this depends on the "Don't fragment" header. If this header is set to 0, the source will simply send packets of a size based on the MTU value of its own interface, so the cached value of the intermediate router r1 will not have any effect. If the "Don't fragment" header is set, IPv4 will behave like IPv6.