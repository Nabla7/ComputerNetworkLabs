The Wireshark capture shows the attempts by the clients to renew their IP leases. Here are the two stages:

\begin{enumerate}[label=\textbf{Stage \arabic*:}]
\item \textbf{Unicast DHCPREQUEST}

Initially, when the lease time is about to expire (typically at 50 percent of the lease duration), the DHCP client attempts to renew its lease with the DHCP server that originally granted it. This is done via unicast, meaning the DHCPREQUEST is sent directly to the known DHCP server's IP address. The client doesn't yet consider that the DHCP server might be down or unreachable.

From our Wireshark capture, we can see multiple unicast DHCP Requests from different clients (10.0.1.2, 10.0.1.3, 10.0.2.2, etc.) to the DHCP server (10.0.1.1 and 10.0.2.1). However, these DHCP Requests are not successful because the server is down. The clients receive ICMP messages indicating that the destination port (the DHCP server) is unreachable.

\item \textbf{Broadcast DHCPREQUEST}

After the initial unicast renewal attempts fail, the client enters the REBINDING state. This occurs when the lease time has reached 87.5 percent of its duration. The client then broadcasts a DHCPREQUEST to the entire network, hoping to reach any operational DHCP server. It no longer tries to reach the original DHCP server exclusively.

In our Wireshark capture, we can see the DHCP Requests are now being sent to the broadcast address (255.255.255.255) from the clients (10.0.1.50, 10.0.2.50, etc.).

\end{enumerate}

\textbf{After Renewal Fails}

If the lease expires and no DHCP server has responded, the client must stop using the network address it was assigned. It will then start the process from the beginning, broadcasting a DHCPDISCOVER message in an attempt to find a DHCP server and obtain a new lease. This is visible in your Wireshark capture as well - the source address is now 0.0.0.0, indicating the client no longer has a valid IP address and is trying to discover a DHCP server to get one.