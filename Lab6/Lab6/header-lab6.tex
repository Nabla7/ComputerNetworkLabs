\usepackage{a4wide}
\usepackage[printonlyused]{acronym}
\usepackage{amsmath}
\usepackage{amssymb}
\usepackage{array}
\usepackage{boxedminipage}
\usepackage{calc}
\usepackage{cite}
\usepackage{color}
\usepackage[inline]{enumitem}
\usepackage{fancyvrb}
\usepackage[draft]{fixme}
\usepackage[T1]{fontenc}
\usepackage{framed}
\usepackage{ifthen}
\usepackage{xifthen}
\usepackage{forloop}
\usepackage[pdftex]{graphicx}
\usepackage{helvet}
\usepackage[colorlinks,linkcolor=black,urlcolor=blue]{hyperref}
\usepackage{listings}
\usepackage{longtable}
\usepackage{ifthen}
\usepackage{listings}
\usepackage{longtable}
\usepackage{multirow}
\usepackage{subfig}
\usepackage{tikz}
\usepackage{titlesec}
\usepackage{verbatim}
\usepackage{xspace}
\usepackage{import}
\usepackage{standalone}
\usepackage{accsupp}
\usepackage{dirtree}

\titleformat{\subsection}{\large\bfseries}{Part \arabic{subsection}.}{0.7em}{}
\setlength{\parindent}{0pt}

\title{
	\textbf{Computer Networks Lab\\ \vspace{1mm} introduced for \vspace{1mm} \\ `Computernetwerken'} \\
		\vspace{2mm}\large{\emph{by Andrei Belogaev}}\\
}

\providecommand{\texroot}{..}
\newcommand{\images}{\texroot/images}

\newcommand{\labdir}{\texroot/Lab\thechapter}

\renewcommand{\chaptername}{Lab}
\newcommand{\stress}[1]{\emph{\textbf{#1}}} 
%\newcommand{\command}[1]{\texttt{#1}\newline} 
\newcommand{\incommand}[1]{\texttt{#1}}
\newcommand{\ipaddr}[1]{\texttt{#1}}

\newcommand{\tracelabel}[2][1]{%
	L\arabic{chapter}-\arabic{Exercisecount}-#1.#2%
}

\newcommand{\tracefile}[2]{traces/L\thechapter-\theExercisecount-#1.#2}

\newcommand{\abstracefile}[2]{\labdir/\tracefile{#1}{#2}}

% optional argument for question number. Format: L{chapter}-{Exercise}-{Question}. Default is question 1.
\newcommand{\file}[2][1]{%
	\texttt{\tracelabel[#1]{#2}}%
}

% Include content of file in traces/ dir. File given as argument has to be present.
\newcommand{\includetrace}[2][]{%
	\ifthenelse{\isempty{#1}}%
	{%
		\renewcommand{\questionNumber}{\thequestioncount}%
	}%
	{%
		\renewcommand{\questionNumber}{#1}%
	}%
	\lstinputlisting{\tracefile{\questionNumber}{#2}}%
}

% Refer to the given file in the traces/ dir. If the file is not present, it is colored red. Blue with hyperlink if present.
\newcommand{\linktrace}[2][]{%
	\ifthenelse{\isempty{#1}}%
	{%
		\renewcommand{\questionNumber}{\thequestioncount}%
	}%
	{%
		\renewcommand{\questionNumber}{#1}%
	}%
	\IfFileExists{\tracefile{\questionNumber}{#2}}{%
		\href{run:\abstracefile{\questionNumber}{#2}}{\tracelabel[\questionNumber]{#2}}%
	}%
	{%
		\textcolor{red}{\tracelabel[\questionNumber]{#2}}%
	}%
}


\newcommand{\curfile}[2]{\texttt{/mnt/L\arabic{chapter}-\arabic{Exercisecount}-\stepcounter{questioncount}\arabic{questioncount}\addtocounter{questioncount}{-1}.#1}} 
\newcommand{\remark}{\includegraphics{\images/remark.pdf}}
\newcommand{\osversion}{Xenial Xerus (16.04.2 LTS)}

\newcommand{\result}[1]{
\begin{verbatim}
	#1 
\end{verbatim}
} 
\newcommand{\prompt}[1]{#1:$\sim$\#}


%%%%%%%%%%%%%%%%%%%%%%%%%%%%%%%%%%%%%%
% exercise environment
%%%%%%%%%%%%%%%%%%%%%%%%%%%%%%%%%%%%%%

\newcounter{Exercisecount}[chapter]
\setcounter{Exercisecount}{0}
\newenvironment{exercise}[1]
{% This is the begin code
\refstepcounter{Exercisecount} {\textbf {Exercise \arabic{Exercisecount}}}: #1
\par
}
{% This is the end code
 }

%%%%%%%%%%%%%%%%%%%%%%%%%%%%%%%%%%%%%%
% question counter environment
%%%%%%%%%%%%%%%%%%%%%%%%%%%%%%%%%%%%%%
\newcounter{questioncount}[Exercisecount]
\setcounter{questioncount}{0}
\newcommand{\questionNumber}{\thequestioncount}

\newcommand{\answerlabel}{L\arabic{chapter}-\arabic{Exercisecount}-\arabic{questioncount}}

\newcommand{\questionlabel}{
	\begin{tikzpicture}[baseline={([yshift=-.8ex]current bounding box.center)}]
 
	\node [fill=shade,rounded corners=5pt]
	{
	\color{white}{\textbf{\href{run:\absanswerfile}{\answerlabel}}}
	};
	\end{tikzpicture}
}

% file relative to lab dir
\newcommand{\answerfile}{solutions/\answerlabel.tex}

% file relative to compile source
\newcommand{\absanswerfile}{\labdir/\answerfile}

\newcommand{\refanswerdir}{../../\labdir}
\newcommand{\refanswerfile}{\refanswerdir/\answerfile}

\newcommand{\includeanswer}{\input{\answerfile}}
\newcommand{\includerefanswer}{
\BeginAccSupp{ActualText=}		% makes text uncopyable
	\subimport*{\refanswerdir/}{\answerfile}
	% \input{\refanswerfile}
\EndAccSupp{}
}

\newcommand*{\scorefile}{scores/\answerlabel.tex}
\newcommand*{\includescore}{\input{\scorefile}}

\newboolean{includeRefAnswer}
\setboolean{includeRefAnswer}{false}
\newcommand{\esolution}{
	{
		\color{blue}
		\par\hrulefill
		\IfFileExists{\answerfile}{%
			\lstset{title=\textcolor{blue}{\lstname}} 
			\par\includeanswer
		}{%
			\par
		}%
%		\par\hrulefill
		{
			\ifthenelse{\boolean{includeRefAnswer}}{%
				\IfFileExists{\refanswerfile}{%
					\color{red}
					\lstset{title=\textcolor{red}{\lstname}} 
					\par\includerefanswer
				}{}
			}{}%
		}
		\par\hrulefill
		\par
	}
}

% display question. optional argument denotes wether an answer to the question is expected (include lines or answer).
\newcommand{\question}[3][true]{%
	\refstepcounter{questioncount}%
	\par \questionlabel #2
	\marginpar{%
		\textcolor{black}{%
			\IfFileExists{\scorefile}{%
				\href{\scorefile}{%
					\color{red}%
					\raisebox{0pt}{\includescore}/ #3%
				}%
			}{%
				\hspace{0.5cm}/#3%
			}%
		}%
	}%
	\ifthenelse{\boolean{#1}}{%
		\esolution%
	}{}%
}

%%%%%%%%%%%%%%%%%%%%%%%%%%%%%%%%%%%%%%
% verbatim solution environment
%%%%%%%%%%%%%%%%%%%%%%%%%%%%%%%%%%%%%%
%\newcounter{linescounter}
%\newenvironment{esolution}[1]
%{
%\answer
%\ifthenelse{\boolean{Solutions}}
%{
%\color{blue}
%
%\hrulefill
%
%\includeanswer
%
%}
%{\setlength{\extrarowheight}{0.75cm} 
%
%\forloop{linescounter}{0}{\value{linescounter}<#1}{\quad\newline\vspace{12pt} \dotfill\newline}
%\comment
%}
%}


% deprecated anwer environment (use command instead)
%\newenvironment{esolution}
%{
%	\answer
%	\ifthenelse{\boolean{Solutions}}
%	{
%	\color{blue}
%	
%	\hrulefill
%	
%	\includeanswer
%	
%	}
%	{\setlength{\extrarowheight}{0.75cm} 
%	\quad\newline\vspace{12pt} \dotfill\newline
%	\comment
%	}
%}
%{
%	\ifthenelse{\boolean{Solutions}}
%	{
%	
%	\hrulefill
%	
%	}
%	{\endcomment}
%}






\DefineVerbatimEnvironment{Verbatim}{Verbatim}
{formatcom=\color{blue},fontfamily=courier,fontseries=b,frame=lines,numbers=left}

\DefineVerbatimEnvironment{cmdblock}{Verbatim}
{gobble=1,frame=leftline,framerule=0.4mm}



%%%%%%%%%%%%%%%%%%%%%%%%%%%%%%%%%%%%%%
% shortcuts
%%%%%%%%%%%%%%%%%%%%%%%%%%%%%%%%%%%%%%
\newcommand{\wifi}{IEEE 802.11\xspace}

%%%%%%%%%%%%%%%%%%%%%%%%%%%%%%%%%%%%%%
% fonts
%%%%%%%%%%%%%%%%%%%%%%%%%%%%%%%%%%%%%%
\renewcommand{\rmdefault}{phv}
\renewcommand{\sfdefault}{phv}

%%%%%%%%%%%%%%%%%%%%%%%%%%%%%%%%%%%%%%
% colors
%%%%%%%%%%%%%%%%%%%%%%%%%%%%%%%%%%%%%%
\definecolor{shade}{HTML}{3877A9}	%light blue shade
\setlength{\parskip}{10pt plus 1pt minus 1pt}

\lstset{ %
basicstyle=\footnotesize,       % the size of the fonts that are used for the code
numbers=left,                   % where to put the line-numbers
numberstyle=\footnotesize,      % the size of the fonts that are used for the line-numbers
stepnumber=2,                   % the step between two line-numbers. If it's 1, each line 
                                % will be numbered
numbersep=5pt,                  % how far the line-numbers are from the code
backgroundcolor=\color{white},  % choose the background color. You must add \usepackage{color}
showspaces=false,               % show spaces adding particular underscores
showstringspaces=false,         % underline spaces within strings
showtabs=false,                 % show tabs within strings adding particular underscores
frame=single,                   % adds a frame around the code
tabsize=2,                      % sets default tabsize to 2 spaces
captionpos=b,                   % sets the caption-position to bottom
breaklines=true,                % sets automatic line breaking
breakatwhitespace=false,        % sets if automatic breaks should only happen at whitespace
title=\textcolor{blue}{\lstname},    % show the filename of files included with \lstinputlisting;
                                % also try caption instead of title
escapeinside={\%*}{*)},         % if you want to add a comment within your code
morekeywords={*,...}            % if you want to add more keywords to the set
}

%%%%%%%%%%%%%%%%%%%%%%%%%%%%%%%%%%%%%%%%%%%%%%%%
%        Fill in your name and group ID        %
%%%%%%%%%%%%%%%%%%%%%%%%%%%%%%%%%%%%%%%%%%%%%%%%
\author{Name 1 \\ Name  2\\ Group groupID}  
%%%%%%%%%%%%%%%%%%%%%%%%%%%%%%%%%%%%%%%%%%%%%%%%



